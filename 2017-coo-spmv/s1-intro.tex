\section{Introduction}
\label{s1-intro}
Applying a discretized operator
in terms of a sparse matrix-vector product (\spmv)
is a heavily-used operation in many scientific applications.
An example are the Krylov subspace methods relying on the \spmv kernel
to generate the Krylov subspaces in which the solutions
to linear and eigenvalue problems are approximated.
At the same time, the \spmv is a frequent bottleneck in complex applications,
as it is notorious for sustaining low fractions of peak processor performance.
This is partly due to the low arithmetic intensity making the 
\spmv kernel memory bound on literally all modern architectures,
the access overhead induced by storing only the nonzero elements in the matrix,
the (in many cases random) access to the input vector.
Given the importance of this building block,
significant effort is spent on finding the best way to store
sparse matrices and optimizing the \spmv kernel for
different nonzero patterns and hardware architectures~\cite{garlandspmv,sellpreport,Liu:2015:CES:2751205.2751209}.
For sparse matrices where the nonzeros are distributed in a very 
structured fashion, it is often possible to derive problem-tailored 
storage formats, like, e.g., the DIA format for matrices with a tridiagonal structure~\cite{garlandspmv}.
A similar situation is given if the pattern is not very structured, but the nonzero elements 
are distributed equally across the rows (each row contains 
a similar number of nonzero elements). 
Most challenging are the 
sparsity patterns that are irregular (no recurring sub-pattern can be identified),
and unbalanced (the distinct rows have a very different number of nonzero elements).
Problems with these characteristics are typical for, e.g., social network representations.
For these irregular problems, standard parallelization strategies, 
like assigning rows to the parallel resources,
inevitably result in heavy load imbalance.
Furthermore, unstructured sparsity patterns often promote 
random memory access to the vector values.

\begin{figure}
\begin{minipage}{\columnwidth}
\lstinputlisting[language=C,caption=,morekeywords={pragma, omp,
parallel,private}]{codes/coo.c}
\end{minipage}
\caption{\coo \spmv kernel design.}
\label{fig:spmv}
\end{figure}

In this paper we present
a GPU implementation 
of the sparse matrix-vector product (Section~\ref{s3-kernel})
that addresses the challenge of overcoming the load imbalance in unstructured
matrices. The kernel is based on the coordinate (COO) format~\cite{garlandspmv}, 
leverages the latest features of the CUDA programming model,
and succeeds in achieving high performance for unstructured matrices.
In a comprehensive evaluation in Section~\ref{s4-experiments} 
we identify the developed kernel as competitive or superior
to the existing routines.
Prior to presenting the new implementation, in Section~\ref{s2-related} we
review
existing efforts for optimizing the sparse matrix vector product
on manycore architectures.
