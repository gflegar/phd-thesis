\section{Summary and Outlook}
\label{s5-conclusion}
We addressed the challenge of overcoming the load-imbalance in the
sparse matrix vector product for irregular matrices.
We developed and implemented an SpMV kernel for GPUs that is based on the COO format.
Using a large collection of test matrices we compared the performance of the new kernel
to the (up to our knowledge) best SpMV kernels available: 
the CSR and hybrid kernels which are part of NVIDIA's cuSPARSE library, the SELL-P and CSR5 kernels
part of the MAGMA-sparse software library, and the CSRI kernel, which balances 
the workload via atomic operations.
We used different metrics to quantify the performance: 
median absolute performance in GFLOPs and its variation,
kernel winning most test cases, and
smallest overhead compared to the best kernel included in the test suite.
For the chosen test suite containing 400 matrices,
the proposed COO-based SpMV performs radically better on irregular matrix
structures, and ultimately wins all considered performance metrics.

In the future we want to focus on multi-GPU architectures
and optimize the developed kernel for hybrid (Multicore+Manycore) execution. 
Furthermore, we are convinced that the strategies reducing the impact of global
write conflicts via warp-vote functions and introducing additional operations
are also applicable to other computational problems of irregular nature.
