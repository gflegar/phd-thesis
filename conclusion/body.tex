\section{Designing Scientific Software for Sparse Computations [OPTIONAL]}
\subsection{Introduction}
TODO: Lessons learned by writing scientific software.
\subsection{Linear Systems}
TODO: Derivation of the linear system solver operator $S_A$.
\subsection{Matrices}
TODO: Derivation of the matrix application linear operator $L_A$.
\subsection{Preconditioners}
TODO: Derivation of the preconditioner linear operator $P^\Pi_A$.
\subsection{Linear Operators --- Towards a Generic Interface for Sparse
            Computations}
TODO: The linear operator and linear operator factory abstractions as the basis
      of the library for sparse computations.
      If I feel like it, maybe I add the new ideas I have about having subspaces
      as first-class citizens in the library, which I feel are a better
      abstraction than vectors we're usually talking about - both for linear
      solvers, and for getting towards eigensolvers. Just to write it down
      somewhere, and maybe someone adds them to Ginkgo.
\subsection{Numerical Methods}
TODO: The issue of numerics and convergence - rounding errors affecting
matrices, preconditioners and solvers, limited convergence affecting solvers and
some preconditioners. The difference between the interface (what we want to
compute) and the numerical method (what we are actually computing), i.e. the
response to ``a Krylov solver is not a linear operator'' comment.

\subsection{Ginkgo: A High Performance Linear Operator Library}
TODO: The Ginkgo library implementing the things discussed above. Open and
modern development practices.

\section{Conclusions and Open Research Directions}
TODO: Adaptive precision algorithms, advanced preconditioners (ILU, threshold
ILU), multi-GPU algorithms, heterogeneous algorithms, distributed algorithms,
eigensolvers, extending the library design to incorporate these features.
