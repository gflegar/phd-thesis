\section{Conclusions and Open Research Directions}

This work focused on GPU acceleration of components for the iterative solution
of linear systems, and showed that significant performance improvements on
modern hardware can be obtained even with basic, well-studied building
blocks.

Part~\ref{pt:matrix-vector} began the discussion by focusing on the most time
consuming and difficult to parallelize operation: the sparse matrix-vector
product. Even though a variety of advanced sparse matrix formats and
accompanying matrix-vector product algorithms were recently proposed,
Chapter~\ref{ch:2017-csr-spmv} showed that most of them exhibit corner cases
where they consume significantly more memory than the standard formats, or
achieve far lower matrix-vector product performance than standard
implementations. Furthermore, the majority of existing software packages relies
on one of the standard formats to realize high performance implementations of
other operations, and as means of managing software complexity and developer
burden. Thus, while application-specific formats can undoubtedly outperform the
conventional ones, they should only be developed in case the potential
improvements over the formats provided by the underlying general-purpose library
(accounting for the necessary format conversions required for interfacing with
other parts of the library) outweigh their development cost.

Chapter~\ref{ch:2017-csr-spmv} focused on reducing the improvement potential ---
and the need to invest resources in the development of application-specific
formats --- by optimizing the corner cases of the most widely used CSR format.
These optimizations are enabled by advancements in accelerator technology, which
recently started offering full support for atomic operations, and effectively
deal with the issue of imbalanced sparsity patterns.  While the new algorithm
cannot compete with the standard variant on regular sparsity patterns, the
ultimate matrix-vector product algorithm can be composed by coupling the two
algorithms with a simple heuristic that predicts the winner.

Chapter~\ref{ch:2017-coo-spmv} continued the development of the sparse
matrix-vector product by identifying the COO format as an alternative
general-purpose format for GPUs. Similarly to the improved CSR algorithm, the
new algorithm for COO is highly efficient and does not suffer from extremely
unfavorable sparsity patterns. However, its higher minimum and average
performance make it a better default choice than CSR. Ultimately, reasonable
use-cases can be found for both options: the improved CSR algorithm can be used
in conjunction with software that uses CSR historically or because the
efficiency of other operations depends on CSR, while COO can be adopted as the
default choice for new software whose performance does not depend on CSR.

While most sparse matrix-vector product algorithms are focused on large
problems that utilize the entire GPU (processor group),
Chapter~\ref{ch:2017-batched-spmv} explored the underdeveloped case of smaller
problems suitable for individual streaming multiprocessors (single processors).
It showed that this case can be implemented more efficiently by slightly
modifying standard algorithms to make better use of the available memory
hierarchy.

New findings presented in this part  can be used in the development of more
specialized matrix formats. For example, the ideas from the COO algorithm are
currently being used for the development of an improved hybrid matrix
format~\cite{hybrid}. The success of these synchronization-free load-balanced
algorithms on a single GPU should also prove useful for the development of
sparse matrix-vector product algorithms that utilize computational resources of
the entire node in unison, as synchronization and load-balancing penalties
become more pronounced on higher levels of the hardware hierarchy. Finally, the
simple heuristic used to select between the two CSR algorithms is also a small
contribution to research in automatic sparse matrix format selection. This
research area focuses on selecting the best format based on the properties of
the matrix, by using certain policies (often based on machine learning
algorithms) to decide among several available
formats~\cite{clspmv,gpu-selection}, or even assemble a matrix-vector product
implementation from a pool of potential optimizations~\cite{elafrou}.

Part~\ref{pt:preconditioning} explored the potential of block-Jacobi
preconditioning on highly parallel GPU hardware. Even though this relatively
simple preconditioner usually features lower convergence rate improvement than
the more widely used ILU-based preconditioners, this part of the thesis showed
that problems with inherent block structure can greatly benefit from it. The
effectiveness of block-Jacobi can be attributed to its inherent parallelism, as
each block can be processed independently. The first step towards a high
performance implementation consists of mapping the blocks to the appropriate
level of the hardware hierarchy. By assigning each block to a single vector
unit, taking advantage of increased register counts and warp shuffle
instructions available on recent hardware, and replacing conventional pivoting
strategies with implicit pivoting, the resulting algorithms are able to
outperform equivalent functionality available in vendor libraries. First such
algorithm, the block-Jacobi preconditioner based on Gauss-Jordan elimination,
was presented in Chapter~\ref{ch:2017-gje-block-jacobi}. The algorithm inverts
the diaginal blocks during preconditioner generation, which means that the
application stage can be realized as a sequence of highly parallel dense
matrix-vector products.

However, as discussed in Section~\ref{introduction:sec:linear-systems}, the
solution of a linear system via matrix inversion can result in numerical
instability. Chapter~\ref{ch:2017-gh-block-jacobi} addressed these concerns by
showing that in practice, there is no difference in preconditioner quality
when using inversion-based, as opposed to factorization-based approach. It also
revisited the unconventional Gauss-Huard method for the solution of linear
systems, and revealed that this method can be superior to the inversion-based
algorithm if only few iterations of the Krylov solver are needed. Finally,
Chapter~\ref{ch:2017-lu-block-jacobi} compared the Gauss-Huard solver
with the standard LU factorization algorithm, and showed that, provided the
conventional ``lazy'' triangular solve algorithm is replaced with the ``eager''
variant, LU factorization can outperform Gauss-Huard decomposition.

The contributions presented in this part constitute only a small sample of
recent developments in preconditioning techniques. Some of these developments
include new highly parallel methods for solving triangular
systems~\cite{triangular-solve, isai, triangular-iterative}, and parallel
generation of threshold ILU preconditioners~\cite{ilut, ilut-gpu}. As a direct
extension of the block-Jacobi preconditioner presented here, future research can
explore the effectiveness of other preconditioners based on relaxation methods.

All algorithms presented in this part are also related to a broader topic of
``batched'' routines, which apply the same operation on a sequence of small
problems. While there are recent proposals for a standardized batched BLAS
interface~\cite{batched-blas}, it is still unclear if this effort will result in
wider adoption, as there are major issues with the current proposal. One such
issue concerns the data format used to store the blocks. Since distinct
architectures and applications require specific storage schemes (\eg, one
parameter batch is shared, another stored as a contiguous sequence, or scattered
throughout the memory), covering all the options leads to an exponential number
of interface functions with an unmanageable amount of parameters. Another issue
arises from the implicit synchronization between two consecutive batched routine
calls. While there are no dependencies between distinct problems in the batch,
the entire batch is still synchronized, as each batched call is essentially a
(parallel) loop over all problem instances. While this can be partially
alleviated by implementing the routines in terms of a set of jobs submitted to a
job scheduling system, relying on the existence of such a system is not always
an option. For example, this is the case for block-Jacobi preconditioning
presented in this work, where the ``batch'' of problems is distributed on the
GPU --- where scheduling systems are not commonly used --- and there is
additional code needed for preprocessing and postprocessing of the problem data
as part of the same GPU kernel. Ultimately, the pragmatic solution might be to
depart from the idea of ``batched'' routines, and instead build libraries that
provide BLAS-like functionality for various levels of the hardware hierarchy.
Essentially, the responsibility of building the outer parallel loop would be
left to the user, while the need for complicated parameter lists would be
removed, the number of interface variations greatly reduced, and implicit
synchronizaitons of unrelated problems avoided. While such libraries are still
uncommon, NVIDIA has recently taken a step towards this direction with its
CUTLASS~\cite{cutlass} library, which provides matrix-matrix multiply
implementations (GEMM) for various levels of the GPU hierarchy.

As a result of recent trends in HPC and the emergence of hardware support for
low-precision computing, Part~\ref{pt:adaptive} evaluated the potential of these
techniques in combination with preconditioning.
Chapter~\ref{ch:2017-adaptive-block-jacobi} analysed the theoretical aspects and
the effect on convergence rate when lower precision storage is used in the
block-Jacobi preconditioner. The success of the technique used is based on
several observations: 1) since preconditioner application is a memory bandwidth
bound operation, performance improvements are possible by reducing the precision
of data stored in memory, and not the precision of computations; 2) a
preconditioner is already an approximation of the original system matrix, so the
storage accuracy does not need to be higher than the accuracy of the
approximation; and 3) storage accuracy cannot be reduced blindly, but has to be
carefully tuned to the numerical properties of the problem. Practical
experiments showed that the combination of these techniques successfully reduces
total storage while preserving preconditioner quality, and, according to a
theoretical energy model, can reduce energy consumption of the complete Krylov
solver. While theoretical analysis alone does not guarantee the feasibility of
practical implementations, new research has demonstrated the effectiveness of
adaptive block-Jacobi preconditioning~\cite{adaptive-jacobi-gpu}, and produced a
first GPU implementation of this preconditioner, which is now available in the
Ginkgo library~\cite{ginkgo}.

As the adaptive precision block-Jacobi represents a pioneering idea in
preconditioner design, it may be possible to enhance other preconditioner using
similar techniques. Adaptive block-Jacobi preconditioning itself may also be
improved further by using non-conventional storage formats instead of the
standard IEEE floating point types~\cite{adaptive-jacobi-gpu}. In the bigger
picture of iterative methods, adaptive block-Jacobi preconditioning is only a
single example of recent research in low-precision methods. In addition to the
well-known iterative refinement~\cite{higham-ir}, other examples include the
adaptive versions of the Jacobi and PageRank relaxation
methods~\cite{jacobi,jacobi-modular,pagerank} and the development of the modular
precision storage format, which enables efficient access to the same array of
floating point values in multiple
precisions~\cite{jacobi-modular,pagerank,anzt-ir}. Orthogonal to
developments in unconventional storage formats, the potential of performing
computations in such formats is also being explored. While this research rarely
results in performance improvements on conventional hardware, it has the
potential to influence the design of future hardware by demonstrating
its effectiveness on simulated systems~\cite{floatx, flexfloat}.

% With the end of Dennard scaling, and potential failure of Moore's law in the
% near future, current HPC landscape is experiencing disrupting changes. This
% work is a product of these changes, and demonstrates that 
