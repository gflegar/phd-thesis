\section{Introduction}
\label{2017-gje-block-jacobi:sec:s1-intro}

Solving large, sparse-linear systems of equations is a prevailing problem
in scientific and engineering applications that involve the discretization of
partial differential equations (PDEs). A standard approach to tackle these
problems combines a Krylov method with a preconditioner that accelerates the
iterative solution process~\cite{saad}. In the context of high-performance
computing, the efficiency of the preconditioner depends on the parallel
scalability of both the preconditioner generation (prior to the iterative solve)
and the preconditioner application (at each step of the iterative solve).

Using preconditioners based on Jacobi (diagonal scaling) and block-Jacobi
typically renders moderate improvements to the convergence of the iterative
solver~\cite{saad}. These acceleration techniques are nevertheless reasonably
attractive as block-diagonal scaling introduces very small computational
overhead to the solver iteration.
Furthermore, the application of a Jacobi-type preconditioner is inherently
parallel and, therefore, highly appealing for massively parallel architectures.

In~\cite{Anzt:2017:BGE:3026937.3026940} we proposed a batched routine for the 
generation of a block-Jacobi preconditioner using the explicit inversion of the
diagonal blocks. Precisely, we designed a variable-size batched routine for
matrix inversion on graphics processing units (GPUs) based on Gauss-Jordan
elimination (GJE)~\cite{Householder}. Furthermore, we introduced an implicit
pivoting strategy in the GJE procedure that replaces row-swapping {with a
	migration of the workload (i.e., operations) to the thread that owns the
	necessary data}, which allowed us to realize the complete inversion process in thread-local
registers. For the block-Jacobi preconditioner generation, the inversion process
needs to be combined with routines that extract the diagonal blocks from the
sparse data structure that stores the coefficient matrix. This extraction step
can be costly, particularly for matrices with an unbalanced nonzero
distribution. In response, we developed an extraction routine that balances
coalescent data access, workload imbalance, and use of shared memory.

In this paper, we extend our previous work in~\cite{Anzt:2017:BGE:3026937.3026940} with new contributions, listed below.
\begin{itemize}
\item We propose a modified version of our variable-size batched GJE (BGJE) inversion routine for GPUs that 
      {can invert} 
      several blocks per warp. This avoids idle CUDA cores and
        operations on dummy values when processing a matrix batch where each
        matrix is less than or equal to ~16 $\times$ 16.
\item We introduce a new variant of the extraction procedure that 
requires a much smaller amount of shared memory.
      This strategy transposes the diagonal blocks at the time they are extracted from
        the sparse {coefficient} matrix,
      inverts the transposed diagonal block, and ultimately writes the inverse of the transpose in transposed {mode}.
      The result provides a functionality equivalent to the original block inversion 
      but reduces the amount
      of shared memory in the block size, utilized during the inversion procedure, from quadratic to linear only.
\item We replace the general sparse matrix--vector multiplication kernel in the preconditioner application with a 
      specialized variant that exploits the block-diagonal structure of the preconditioner matrix. 
      This accelerates the application of the block-Jacobi preconditioner in the iterative solution process.
\end{itemize}
{Our} results revealed that these modifications can render
significant performance improvements, particularly when targeting batches
consisting of small blocks like those appearing in block-Jacobi preconditioning
for problems arising from finite element method (FEM) discretizations.

The rest of the paper is structured as follows. In
Section~\ref{2017-gje-block-jacobi:sec:s2-background} we offer a short review of Jacobi-type iterative
solvers and batched routines for linear algebra. In Section~\ref{2017-gje-block-jacobi:sec:s3-kernel}
we further elaborate on the batched Gauss-Jordan elimination (BGJE) procedure presented
in~\cite{Anzt:2017:BGE:3026937.3026940}, and we describe the batched kernels and
highlight the major improvements in the block-Jacobi generation step, extraction
step, and preconditioner application. In Section~\ref{2017-gje-block-jacobi:s4-experiments} we report on
our extensive evaluation of the new BGJE routine on NVIDIA's K40 and P100 GPUs.
Particularly, we focus on the performance acceleration {produced by} the
modifications of the original BGJE. Finally, in Section~\ref{2017-gje-block-jacobi:sec:s5-summary} we
summarize our contributions and the insights gained from the experimental
evaluation.
