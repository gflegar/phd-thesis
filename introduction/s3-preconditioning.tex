\section{Preconditioning}

As outlined in Section~\ref{introduction:sec:linear-systems}, the convergence
rate of Krylov methods is tied to a function of its spectrum. Thus, if the
original system is replaced with a different system that has the same solution
but better spectral properties, the method will converge in fewer iterations.
This can be achieved by transforming the system into a \emph{preconditioned
system} in one of the following ways:
\begin{itemize}
\item \emph{left preconditioning},
\begin{equation}
    M^{-1}Ax = M^{-1}b;
\end{equation}
\item \emph{right preconditioning},
\begin{equation}
    AM^{-1}y = b,
\end{equation}
where $Mx = y$;
\item \emph{two-sided preconditioning},
\begin{equation}
    M_1^{-1}AM_2^{-1} y = M_1^{-1}b,
\end{equation}
where $M_2x = y$ and $M = M_1 M_2$.
\end{itemize}

To make sure the preconditioned system is easier to solve than the original one,
the \emph{preconditioner} $M$ should be chosen in a way that $M^{-1}A$,
$AM^{-1}$ or $M_1^{-1}AM_2^{-1}$ is better conditioned than $A$, or at least has
fewer extreme eigenvalues. In addition, one needs to compute $M^{-1}b$ and a
series of matrix-vector products $z = M^{-1}Ay$, so $M$ should be chosen in a
way that makes computing $z = M^{-1}w$ easy. Unfortunately, these two
requirements are mutually exclusive. The first one is optimized by setting
$M = A$, resultng in perfectly conditioned system matrix $M^{-1}A = I$, but
the opeartion $z = M^{-1}w = A^{-1}w$ is as difficult as the original problem.
On the other hand, the second is optimized by setting $M = I$, which does
not yield any improvements of the spectral properties. Thus, an effective
preconditioner balances the tradeoffs between the two extremes and provides
moderate improvements of the spectrum, while keeping its structure simple enough
for computing $z = M^{-1}w$ cheaply. Finding good preconditioners is an area of
active research and, while there are no methods which would find perfect ones,
there are several heuristics that generate good preconditioners for certain
types of problems.

One category of heuristics is derived directly from relaxation
methods~\cite{saad}. By setting $G := M^{-1}N$ and $f := M^{-1}b$, the
relaxation method equation~(\ref{introduction:eqn:relaxation}) can be rewritten
as:
\begin{equation}
    x_{k+1} = Gx_{k} + f.
\end{equation}
This is in fact Richardson iteration (see
Table~\ref{introduction:tab:relaxation-methods}) with parameter $\alpha = 1$ for
the system
\begin{equation}
    (I - G)x = f.
\end{equation}
Using the equalities $I - G = I - M^{-1}N = M^{-1}(M - N) = M^{-1}A$ and $f =
M^{-1}b$, it can be rewritten as
\begin{equation}
    M^{-1}Ax = M^{-1}b.
\end{equation}
This shows that every relaxation method is just Richardson iteration on a
preconditioned system, where the preconditioner $M$ is the same matrix that
defines the splitting in Table~\ref{introduction:tab:relaxation-methods}. Thus,
every matrix that defines a relaxation method can also be used as a
preconditioner, defining the Jacobi, Gauss-Seidel, SOR($\omega$) and
SSOR($\omega$) preconditioners and their blocked variants.

Another class of preconditioners is obtained by using the ideas from sparse
direct methods~\cite{saad}. Instead of performing the often expensive full
factorization, one can limit the amount of fill-in to obtain an approximate
factorization $A = F_1 \cdot \ldots \cdot F_k - R$, where $R$ is the residual of
the approximation. The approximate factorization is then used as the
preconditioner $M = F_1 \cdot \ldots \cdot F_k$. These ideas can be combined
with various ways of controlling the fill-in of the factors and result in
families of Incomplete LU (ILU) and Incomplete Cholesky (IC)
preconditioners~\cite{saad, parilut}.

Other preconditioning heuristics include methods for approximating the inverse
$A^{-1}$, using a small number of iterations of another method, or using
problem-specific knowledge directly to construct a suitable preconditioner.
