\section{Linear Systems}
\label{introduction:sec:linear-systems}

This section presents a short overview of various methods used for solving
linear systems. As there is already literature available providing extensive
descriptions and theoretical analyses of the methods, this section only aims at
outlining their general classification and introducing the reader with the main
ideas used to develop them. For an introductory text on an undergraduate level,
readers are referred to the book by Ipsen~\cite{Ipsen2009}, which describes the
basic direct methods and provides the most important parts of their theoretical
analysis. Demmel's graduate-level book~\cite{demmel} presents the same topics in
more detail and also provides a significant amount of material on iterative
methods. Finally, advanced material on the topic can be found in the following
trio: Higham's book~\cite{higham} describes the error analysis of direct methods
and iterative relaxation methods; Duff et al.~\cite{duff} provide an extensive
text on sparse direct methods; and a detailed description of iterative
methods is available in Saad's book~\cite{saad}.

Solving a linear system consists of finding a vector $x \in \mathbb{F}^n$ such
that $Ax = b$ for a known matrix $A \in \mathbb{F}^{n \times n}$ and vector $b
\in \mathbb{F}^n$. Here, $\mathbb{F}$ is either the field of real
($\mathbb{R}$), or the field of complex numbers ($\mathbb{C}$) and $n$ is a
positive integer which determines the size of the system. If the system matrix
is nonsingular, the unique solution is equal to $x = A^{-1}b$~\cite{demmel}.
While a straightforward approach would be to compute the matrix inverse $A^{-1}$
and apply it to $b$, this strategy suffers from numerical instability and
needs a suboptimal amount of floating-point
operations~\cite{demmel,Ipsen2009}. In practice, depending on the numerical
properties of the system matrix, one can choose from several
other methods that do not experience this problem. There are two main approaches
used, resulting in a classification between direct and iterative
methods~\cite{demmel}.

\begin{table}
\begin{center}
\begin{tabular}{lll}
Name & Mathematical description & Supported matrix types \\
\midrule
LU factorization & 
    $\begin{aligned}
        &A = LU\\[-0.4em]
        &L \text{ unit lower triangular}\\[-0.4em]
        &U \text{ upper triangular}
    \end{aligned}$ &
general \\[1.5em]
Cholesky factorization &
    $\begin{aligned}
        &A = LL*\\[-0.4em]
        &L \text{ lower triangular}
    \end{aligned}$ &
symmetric positive definite \\[1em]
LDL factorization &
    $\begin{aligned}
        &A = LDL*\\[-0.4em]
        &L \text{ unit lower triangular}\\[-0.4em]
        &D \text{ diagonal}
    \end{aligned}$ &
symmetric \\[1.5em]
QR factorization &
    $\begin{aligned}
        &A = QR\\[-0.4em]
        &Q \text{ unitary}\\[-0.4em]
        &R \text{ upper triangular}
    \end{aligned}$ &
general 
\end{tabular}
\end{center}
\caption{A table of common direct methods for the solution of linear systems.}
\label{introduction:tab:direct-methods}
\end{table}

Direct methods exploit the fact that systems with matrices of some special
structure are relatively easy to solve. For example, a system with a diagonal
matrix ($A_{ij} = 0$ for $i \neq j$) can be solved by dividing the entries of
$b$ with corresponding diagonal entries of $A$; an upper ($A_{ij} = 0$ for $i <
j$) or lower triangular system ($A_{ij} = 0$ for $i > j$) is easily solved with
forward or backward substitutions, respectively~\cite{demmel,Ipsen2009}; a
unitary system ($A^* A = I$) is solved by multiplying the right hand side with
$A^*$. The idea of direct methods is to factorize the original system matrix
into a product of two or more such matrices:
\begin{align}A &= F_1 \cdot \ldots \cdot F_k\end{align}
and solve a series of systems:
\begin{align}
    F_1 x_1 &= b\\
    F_2 x_2 &= x_1\\
    &\vdots \nonumber \\
    F_k x &= x_{k-1}.
\end{align}
The most popular direct methods are listed in
Table~\ref{introduction:tab:direct-methods}. LU factorization is the most common
form and can be used on all nonsingular matrices. The Cholesky factorization
exists only for symmetric positive definite matrices~\cite{demmel}, while the
LDL factorization relaxes this requirement to symmetric matrices, regardless of
their definiteness. The QR factorization works for general matrices. It provides
better error bounds than the LU factorization and can also be used to solve the
least squares problem\cite{demmel}. On the other hand, it needs more operations
than LU and requires the calculation of square roots. Most direct methods need
to be augmented with a pivoting strategy to ensure the existence and numerical
stability of the factorizations listed above, which includes permuting the rows
and columns of the matrix during the factorization process.  Effectively, this
results in the factorization being done on the permuted matrix $B = P A Q^T$.
Assuming the matrix is stored in full, uncompressed form, all of these methods
require $O(n^3)$ floating point operations to produce the factorization and
$O(n^2)$ operations to solve the system for one right hand side. However, they
have a different constant factor hidden underneath the big-$O$ notation.

Iterative methods produce a sequence $x_0, x_1, x_2, \ldots$ of approximations
of the solution $x$, starting from an initial guess $x_0$. The hope is that
the approximation sequence converges towards $x$ and that the approximation is
good enough after a reasonable amount of iterations. Theoretical analysis only
guarantees convergence for some methods and for matrices with certain
properties. However, iterative methods are still attractive, as they do converge
for many classes of real-world problems, the number of iterations can be tuned
to produce a solution of the desired quality, and, in a lot of cases, a
reasonable quality solution can be obtained faster than with a direct
methods~\cite{saad}.  Another advantage is simpler parallel implementation,
better handling of matrix data directly from the (possibly unconventional)
compressed storage format, and often significantly smaller memory footprint (see
Section~\ref{introduction:sec:sparse}).

\begin{table}
\begin{center}
\begin{tabular}{lll}
Name & Splitting & Supported matrix types \\
\midrule
Richardson & 
    $M = \frac{1}{\alpha}I$ &
    general \\[1em]
Jacobi &
    $M = D$ &
    general \\[1em]
Gauss-Seidel &
    $M = D - L$ &
general \\[1em]
SOR($\omega$) &
    $M = \frac{1}{\omega}(D - \omega L)$ &
general \\[1em]
SSOR($\omega$) &
    $M = \frac{1}{\omega(2 - \omega)}(D - \omega L)D^{-1}(D - \omega U)$ &
symmetric \\[1em]
\end{tabular}
\end{center}
\caption{A table of common direct methods for the solution of linear systems.
Matrix $D$ is the diagonal, $-L$ the strict lower triangle and $-U$ the strict
upper triangle of the system matrix $A$. $\alpha$ and $\omega$ are scalar
values.}
\label{introduction:tab:relaxation-methods}
\end{table}

Relaxation methods are the oldest and the simplest class of iterative methods.
The idea is to split the system matrix $A$ into the sum of two matrices $A = M -
N$, where $M$ is nonsingular and a system with matrix $M$ is relatively easy to
solve. Then, the problem can be rewritten as
\begin{align}
Ax &= b\\
(M - N)x &= b\\
Mx &= Nx + b \\
x &= M^{-1}Nx + M^{-1}b
\end{align}
yielding an iterative method via the recurrence relation
\begin{equation}
    x_{k+1} = M^{-1}Nx_k + M^{-1}b. \label{introduction:eqn:relaxation}
\end{equation}
This class of methods converges for any right hand side $b$ and any initial
guess $x_0$  if and only if the spectral radius of matrix $M^{-1}N$ is strictly
less than $1$~\cite{demmel,saad}.
Table~\ref{introduction:tab:relaxation-methods} lists the most common
relaxation methods, together with matrix $M$, which defines the splitting.
Thus, the properties required to fulfill the spectral radius condition
differ among the methods, and depend on the properties of the system matrix
and the choice of the open parameters $\alpha$ and
$\omega$~\cite{barrettemplates}. All of these methods can be transformed into
their blocked variant by replacing the diagonal $D$ with the block-diagonal, the
strict lower triangle $-L$ with the strict lower block-triangle and the upper
triangle $-U$ with the strict upper block-triangle of the system matrix $A$,
which can significantly increase the convergence rate of the solver~\cite{saad}.
Since each iteration consists of several matrix-vector products and vector
operations, the complexity of the method is $O(n^2 m)$, where $m$ is the number
of iterations required to achieve a good enough approximation. Thus, speedups
over direct methods are possible if $m$ is significantly smaller than $n$.

\begin{figure}
\begin{center}
\begin{tabular}{|l|}
\hline
\\Initialize $x_0, r_0 := b-Ax_0, p_0 := r_0, \tau_0 := r_{0}^* r_{0}^{}$
\\ $k := 0$                                                
\\ {\bf while} not converged
\\ ~~~ $q_{k+1}:=Ap_{k}$                        
\\ ~~~ $\eta_k:=p_{k}^*q_{k+1}^{}$    
\\ ~~~ $\alpha_k:=\tau_k/\eta_k$    
\\ ~~~ $x_{k+1}:=x_k+\alpha_k p_{k}$           
\\ ~~~ $r_{k+1}:=r_k-\alpha_k q_{k+1}$            
\\ ~~~ $\tau_{k+1}:= r_{k+1}^* r_{k+1}^{}$  
\\ ~~~ $\beta_{k+1}:=\tau_{k+1}/\tau_{k}$  
\\ ~~~ $p_{k+1}:= r_{k+1} + \beta_{k+1} p_k$ 

\\ ~~~ $k:=k+1$   
\\ {\bf endwhile}  
\\\hline
\end{tabular}
\end{center}
\caption{A pseudocode of the Conjugate Gradient Krylov method.}
\label{introduction:fig:cg}
\end{figure}

A relatively newer, and usually more effective class of iterative methods are
methods based on Krylov subspaces. The motivation for these methods stems from
the Cayley-Hamilton theorem, which states that every square matrix $A$ satisfies
its own characteristic equation $k_A(\lambda) = 0$ (\ie $k_A(A) = 0$), where
$k_A$ is the characteristic polynomial $k_A(\lambda) := \det(\lambda I - A) =
\alpha_0 + \alpha_1 \lambda + \ldots + \alpha_n \lambda^n$ of matrix $A$.
Multiplying the equation with the solution $x$ of the linear system results in
the following formula:
\begin{align}
    k_A(A)x &= 0\\
    \alpha_0 x + \alpha_1 Ax + \ldots + \alpha_n A^n x &= 0\\
    \alpha_0 x + \alpha_1 b + \ldots + \alpha_n A^{n-1} b &= 0\\
    x &= -\frac{1}{\alpha_0}(\alpha_1 b + \ldots + \alpha_n A^{n-1}b),
\end{align}
where the last equation holds since $A$ is non singular, \ie $\alpha_0 = k_A(0)
= \det(A) \neq 0$. Thus, the solution is in one of the Krylov subspaces
$\mathcal{K}_{A, b}^m := \mspan\{b, Ab, \ldots, A^{m-1}b\}$, $m = 1, \ldots, n$.
In practice, finding the coefficients $\alpha_i$ of the characteristic
polynomial is far more difficult than solving the system. Instead, practical
Krylov methods construct a series of subspaces $\mathcal{K}_{A,b}^m$ and find a
projection (orthogonal or oblique) of the solution $x$ onto that subspace. By
using a clever definition of the inner product, this projection can be
obtained without knowing $x$ itself~\cite{demmel,saad}.

If one of the Krylov subspaces is invariant for $A$, \ie $A \mathcal{K}_{A,
b}^m \subseteq \mathcal{K}_{A, b}^m$, then the sequence folds onto itself
($\mathcal{K}_{A, b}^{m + 1} = \mathcal{K}_{A, b}^m \cup A \mathcal{K}_{A, b}^m
= \mathcal{K}_{A, b}^m$) and the exact solution is found after $m$ steps in
$\mathcal{K}_{A, b}^m$. Finding an invariant subspace soon is the hope of Krylov
subspace-based methods, since in that case, only $m$ multiplications with the
system matrix are needed. Even if the sequence does not fold onto itself soon
enough, the hope is that the orthogonal projection of the solution $x$ to the
subspaces $\mathcal{K}_{A, b}^m$ is close enough to $x$, so that the method
finds this solution. Similarly to relaxation methods, the complexity of Krylov
methods is $O(n^2 m)$. Usually, the number of iterations $m$ needed is much
smaller than required by relaxation methods, and, assuming exact arithmetic and
no breakdowns in the orthogonalization process, $m$ is bounded by the size of
the system $n$.  Another appealing property of Krylov subspace-based methods is
the fact that the system matrix is used indirectly, as part of the Krylov
subspace construction, and to define the inner product. This is especially
appealing for a prospective software library designer, as the only operations
where the system matrix is required is its application to a vector (\ie a
matrix-vector product).  Thus, different matrix storage formats and
corresponding matrix-vector product implementations can be easily swapped and
used with the same implementation of the Krylov method.

Figure~\ref{introduction:fig:cg} shows the pseudocode of the Conjugate Gradient
(CG) Krylov method, suitable for symmetric positive definite matrices. The
information about the current Krylov subspace is stored implicitly as part of
the auxiliary vectors $p$, $q$ and $r$. The main components of the method can
clearly be seen: the matrix-vector product used to construct the next vector in
the subspace; and vector operations used to orthogonalize the subspace and
construct the projection of $x$ onto that subspace. The convergence of the
method depends on the spectral properties of the system matrix, with the method
converging by a factor of $(\sqrt{\kappa_2(A)} - 1)/(\sqrt{\kappa_2(A)} + 1)$
per iteration, where $\kappa_2(A)$ is the spectral condition number of
$A$~\cite{barrettemplates,demmel,saad}. Other Krylov methods contain similar
components, with some of them requiring an implementation of conjugate
matrix-vector product ($y = A^*x$) as well. Similarly to CG, their theoretical
convergence can usually be bound by some polynomial of the system matrix'
spectrum. The pseudocode for those methods, as well as their derivation and
theoretical analysis can be found in other
literature~\cite{barrettemplates,demmel,saad}.

The last iterative method discussed in this section is iterative refinement.
This method is not a standalone method, but can be used to improve the accuracy
of other methods. A coarse, less accurate method for solving $Ax = b$ produces a
result $\tilde{x}_0 = x + e$, where $e$ is the error in the solution.  The error
$e$ can be approximated by solving a new system $Ac = r_0$ using the coarse
method, where $r_0 = b - A\tilde{x}_0$, obtaining $c = A^{-1}r_0 = A^{-1}b -
\tilde{x}_0 = x - \tilde{x}_0 = -e$.  $c$ can be used to correct the solution
$\tilde{x}_0$, since $x = \tilde{x}_0 + c$. However, as $c$ is only approximated
using the coarse method, $\tilde{c} = c + e_c$ is obtained instead of $c$. Thus,
the corrected solution is actually $\tilde{x}_1 = \tilde{x}_0 + \tilde{c} = x +
e_c$. Nevertheless, as long as the residual $r_0$ and the upate $\tilde{x}_1$ is
computed more accurately than the solution of the system, the new error $e_c$
will be several orders of magnitude smaller than $e$~\cite{demmel,saad}. The
process can be repeated iteratively to decrease the error further. Iterative
refinement is usually used as a way to obtain a solution
better than the working precision of the coarse method, either by 1) using a
lower precision arithmetic in the coarse method to accelerate the solution
process~\cite{higham-ir,anzt-ir}, or 2) by using non-IEEE compliant or software
defined arithmetic for residual calculation and solution updates, resulting in a
more accurate solution than possible using the standard floating point
types~\cite{demmel}.

For completeness, it is worth mentioning there are more advanced methods for
solving linear systems, which yield significant performance improvements in some
special cases, or even enable solving problems which are otherwise not solvable
via standard techniques. Notably, these are the multigrid and domain
decomposition methods. However, these methods are not in the scope of this
thesis, so the interested reader is referred to other literature describing
them~\cite{demmel,saad}.
