\section{Numerical Methods in High Performance Computing}

While previous sections mostly focused on numerical aspects of various methods
for the solution of linear systems, this section briefly outlines additional
considerations that have to be taken into account when designing high
performance software for current hardware.

The defining characteristic of today's systems is the discrepancy between
processor and memory speed, with most systems being able to perform between 10
and 100 operations for every byte fetched from memory. To resolve the issue,
systems are designed with a hierarchy of increasingly smaller, faster, and more
expensive memories placed between the main memory and the processor. The idea is
to hide the slow main memory by placing often-accessed data into these faster
memories, which is either done automatically by the hardware (\emph{cache}) or
manually by the programmer (\emph{scratchpad memory}). Multiple memory- and
processor modules are connected together to form \emph{nodes}. The memory
modules are often presented as a single unified memory. Nevertheless, depending
on the node configuration, a group of processors can exhibit lower
bandwidth towards some memory modules (\eg NUMA configurations and
heterogeneous systems with Unified Memory enabled), or even have no access
to them (\eg heterogeneous systems with Unified Memory disabled).
State-of-the-art processors can roughly be divided into two categories:
\begin{itemize}
\item \emph{general purpose processors}, which are installed in 1--4 groups (\ie
\emph{sockets}) of 2--30 processors (\ie \emph{cores}) per node, with each one
able to perform 8--32 operations (combined into vector instructions) in
parallel; and
\item \emph{accelerators}, which are installed in 0--16 groups (\eg \emph{GPUs})
of 1--80 processors (\eg \emph{Streaming Multiprocessors}) per node, with each
one able to perform 64--196 operations (combined into vector instructions) in
parallel.
\end{itemize}
Compared to accelerators, general purpose processors usually operate on higher
frequencies and expose less parallelism \ha{also: less latency, more complex
instruction set} \gf{Latency is a direct consequence of higher frequency. More
complex instruction set is only true if you compare the combination of
NVIDIA/AMD GPUs and x86 CPUs, but not in general, and is generally not
something considered when designing the methods, so why bother discussing it?
There are also other differences, like prefetching, but the point is not to give
an exhaustive comparison of CPUs vs accelerators.}, but feature higher energy
cost per operation ratio. As a result, new systems follow a trend where an
increasingly larger proportion of total compute power is supplied by the
accelerators, while the general purpose processors are used for managing I/O
devices, networking, copying memory and orchestrating computation. As a final
layer of the hierarchy, large systems are formed by connecting multiple nodes
into a network (\ie \emph{cluster}). The largest systems contain between 1,000
and 100,000 nodes,
% abci: 1,088
% sierra: 4,320
% summit: 4,608
% daint: 5,272
% supermuc: 6,480
% tianhe: 16,000
% titan: 18,688
% trinity: 19,420
% sunway: 40,960
% sequoia: 98,304
with the number converging towards the lower end, due to the current trend of
reducing the total number of nodes by using more powerful ``\emph{fat}''
nodes~\cite{top500}.

Direct methods for dense systems offer a relatively straightforward mapping to
modern hardware. The volume of data is a lower degree polynomial
($O(n^2)$) than the amount of computation ($O(n^3)$) needed to perform the
factorization, so there are plenty of opportunities for data reuse and effective
utilization of caches and scratchpad memories. In addition, the regular
structure of these problems implies that the amount of computation needed to
process each row, column or block of the system matrix is roughly constant, so
the work can be easily distributed evenly among processors.

On the other hand, direct and iterative methods for sparse systems suffer from
more serious issues. The total data volume, $O(n_{nz})$, where $n_{nz}$ is
the total number of nonzeros in the system, is usually significantly closer to
the total amount of computation, which is $O(n_{nz} m)$ for iterative
methods and ranges between $O(n_{nz})$ and $O(n^3)$ for direct methods,
resulting in far less opportunities for data reuse. Iterative methods in
particular proceed in a sequence of iterations that cannot be combined, and each
one requires the complete problem data. There are ways, however, to slightly
alleviate the issue by processing multiple systems that have the same system
matrix at the same time, or adding multiple vectors to the Krylov subspace in
each iteration~\cite{ca-gmres}. Work distribution is another issue for both
direct and iterative methods. This is a direct consequence of often imbalanced
distribution of nonzeros in the system matrix, which hampers the development of
efficient building blocks, such as the matrix-vector product and factorization
algorithms. Especially difficult are the algorithms for triangular solves since,
depending on the sparsity pattern of the matrix, they can exhibit virtually no
parallelization potential~\cite{triangular-solves}. These methods are often the
Achilles' heel of accelerator-focused systems, as load balancing and low
parallelization potential present significant difficulties on highly-parallel
hardware.  As a result of these issues, state-of-the art methods for sparse
systems are limited by the memory bandwidth and only use a fraction of the
processing power available in today's systems~\cite{top500}.
