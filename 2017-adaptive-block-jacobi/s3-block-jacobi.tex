\section{Block-Jacobi Preconditioning} \label{2017-adaptive-block-jacobi:sec:jacobi}

\newcommand{\hD}{E}

The Jacobi  method splits the coefficient matrix as $A=L+D+U$, with a diagonal
matrix $D=(\{a_{ii}\})$, a lower triangular factor $L=(\{a_{ij}~:~i>j\})$, and
an upper triangular factor $U=(\{a_{ij}~:~i<j\})$. The block-Jacobi variant is
an extension that gathers the diagonal blocks of $A$ into
$D=(D_1,D_2,\ldots,D_N)$, with $D_i\in \R^{m_i \times m_i}$, $i=1,2,\ldots,N$,
and $n=\sum_{i=1}^Nm_i$.
The remaining elements of $A$ are then partitioned into matrices $L$ and $U$
such that $L$ contains the elements below the diagonal blocks in $D$, while $U$
contains those  above them~\cite{gje}. The block-Jacobi method is well defined
if all diagonal blocks are nonsingular. The resulting preconditioner, $M=D$,
is particularly effective if the blocks succeed in reflecting the nonzero
structure of the coefficient matrix,~$A$. Fortunately, this is the case for many
linear systems that, for example, exhibit some inherent block structure because
they arise from a finite element discretization of a partial differential
equation (PDE)
~\cite{gje}.

There are several strategies to integrate a block-Jacobi preconditioner into
an iterative solver like CG. In this work, we adopt an approach that
explicitly computes the block-inverse matrix,
$D^{-1}=(D_1^{-1},D_2^{-1},\ldots,D_N^{-1}) =(\hD_1,\hD_2,\ldots,\hD_N)$, before
the iterative solution process commences, and then applies the preconditioner in
terms of a dense matrix-vector multiplication (\gemv) per inverse block 
$\hD_i$~\cite{ANZT2018}.
Note that \gemv is still a memory bandwidth-bound kernel, independent of the 
block
size. In practice, this strategy shows numerical stability similar to the
conventional alternative that computes the LU factorization (with partial
pivoting)~\cite{GVL3} of each block ($D_i=L_iU_i$) and then applies the
preconditioner using two triangular solvers (per factorized block). By 
comparison,
the \gemv kernel is highly parallel, while the triangular solves offer only 
limited 
parallelism.

