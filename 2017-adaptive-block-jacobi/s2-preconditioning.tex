\section{Reduced Precision Preconditioning in the PCG Method}
\label{2017-adaptive-block-jacobi:sec:cg}

\subsection{Brief review}

\newcommand{\fpd}{\mbox{\rm fp64}\xspace}
\newcommand{\fps}{\mbox{\rm fp32}\xspace}
\newcommand{\fph}{\mbox{\rm fp16}\xspace}
\newcommand{\fpv}{\mbox{\rm fpxx}\xspace}
\newcommand{\inti}{\mbox{\rm int32}\xspace}

Figure~\ref{2017-adaptive-block-jacobi:fig:pcg} shows the PCG method for the solution of the linear system
$Ax=b$; where the coefficient matrix, $A \in \Rnn$, is SPD and sparse with $n_z$
nonzero entries; $b \in \Rn$ is the right-hand side; and
$x \in \Rn$ is the sought-after solution. The most challenging operations in
this algorithm are the computation of the preconditioner (before the iteration
commences), the computation of the sparse matrix-vector product (\spmv) (at 
each iteration), 
and the preconditioner application (at each iteration). The
remaining operations are scalar computations or simple vector kernels like
the dot product (\dotp) and \axpy-type vector updates~\cite{barrettemplates}.

\begin{figure}[t]
{\small
\begin{center}
\begin{tabular}{|l|}
\hline
  $A \rightarrow M$  
\\Initialize $x_0, p_0, r_0 := b-Ax_0, \tau_0 := \parallel r_{0}\parallel_2, 
\gamma_0$
\\ $k := 0$                                                
\\ {\bf while} $(\tau_{k} > \tau_{\max})$   
\\ ~~~ $q_{k+1}:=Ap_{k}$                        
\\ ~~~ $\eta_k:=p_{k}^Tq_{k+1}$    
\\ ~~~ $\alpha_k:=\gamma_k/\eta_k$    
\\ ~~~ $x_{k+1}:=x_k+\alpha_k p_{k}$           
\\ ~~~ $r_{k+1}:=r_k-\alpha_k q_{k+1}$            
\\ ~~~ $\tau_{k+1}:=\parallel r_{k+1}\parallel_2 $  
\\ ~~~ $z_{k+1}:=M^{-1} r_{k+1}$            
\\ ~~~ $\gamma_{k+1}:= r_{k+1}^T z_{k+1}$    
\\ ~~~ $\beta_{k+1}:=\gamma_{k+1}/\gamma_{k}$  
\\ ~~~ $p_{k+1}:= z_{k+1} + \beta_{k+1} p_k$ 

\\ ~~~ $k:=k+1$   
\\ {\bf endwhile}  
\\\hline
\end{tabular}
\end{center}
}
\caption
[Mathematical formulation of the PCG method]
{Mathematical formulation of the PCG method. Here, $\tau_{\max}$ is
the relative residual stopping criterion.}
\label{2017-adaptive-block-jacobi:fig:pcg}
\end{figure}

In the PCG method, the \dotp operations present a
one-to-one ratio
of FLOPs to memory accesses (MEMOPS), and the \axpy-type operations present a 
two-to-three ratio of FLOPs to MEMOPS, which clearly identifies these 
operations as 
memory bandwidth-bound kernels. For simplicity, moving forward we make no 
distinction between 
the cost of reading a number and the cost of writing a number. Assuming the 
sparse 
matrix is stored in compressed sparse row (CSR) format~\cite{saad}---and is 
using 64~bits for double precision numbers/values (\fpd) and 32~bits for 
integers/indices (\inti)---the 
ratio of FLOPs:MEMOPS for \spmv is $2n_z/((n+n_z) \cdot \fpd + (n+n_z) \cdot
\inti)$. As a consequence,
this operation is also memory bounded.
An analysis of the operations using the preconditioner is provided later
in this section.

\subsection{Orthogonality-Preserving Mixed Precision Preconditioning}

In general, using a reduced precision preconditioner (i.e., 32-bit or 16-bit 
arithmetic) 
instead of ``full'' 64-bit, double precision arithmetic
requires a careful consideration of the numerical effects. In this section we 
discuss how our preconditioning strategy results in a constant preconditioning
operator. This preserves the orthogonality of the 
generated Krylov search directions, 
and therefore allows us to employ the standard CG solver based on the 
Fletcher-Reeves orthogonalization instead of the flexible CG based on the 
Polak-Ribi\`{e}re formula.

The PCG method
presented in Figure~\ref{2017-adaptive-block-jacobi:fig:pcg} assumes that the preconditioner is a constant
operator acting on the input vector, $r=r_{k+1}$, as
$z=z_{k+1}:=M^{-1}r$~\cite{saad}. In this case, $r_k^Tz_{k+1}=0$, that is to 
say, the
orthogonality with respect to the previous residual is preserved. Strictly
speaking, even when using double precision, the preconditioner application
introduces some rounding error so that the computed operator satisfies
$z=M^{-1}r + \mathcal{O}(\varepsilon_d)$, where $\varepsilon_d$ stands for \fpd
machine precision. Hence, a preconditioner in double precision can also
have an impact on the orthogonality. However, as the effects are in the order of
the approximation accuracy, the non-consistency of the preconditioning operator
is typically disregarded in practice.

In contrast, when applying a preconditioner in less than double precision, 
this issue becomes more relevant, because the rounding
error now grows to $z=M^{-1}r + \mathcal{O}(\varepsilon_r)$, where
$\varepsilon_r$ is the machine precision of the reduced format. As a result, the
orthogonality error increases to $\varepsilon_r$, which becomes relevant if
convergence beyond $\varepsilon_r$ is desired.

A straightforward workaround is to introduce an additional orthogonalization
step to account for the loss in orthogonality. Concretely, replacing the
Fletcher-Reeves formula from Figure~\ref{2017-adaptive-block-jacobi:fig:pcg},
\begin{align}
\beta_{k}:=\gamma_{k+1}/\gamma_{k}=\frac{r_{k+1}^T z_{k+1}}{r_{k}^T z_{k}},
\label{2017-adaptive-block-jacobi:eq:fr}
\end{align}
with the Polak-Ribi\`{e}re formula,
\begin{align}
\beta_{k}:=\frac{(r_{k+1}-r_k)^T z_{k+1}}{r_{k}^T z_{k}},
\label{2017-adaptive-block-jacobi:eq:pr}
\end{align}
naturally accounts for $z_{k+1}$ losing orthogonality with respect to the
previous residual, $r_k$~\cite{doi:10.1137/S1064827597323415}. Compared
with the original formulation of the CG method, this
orthogonality-preserving ``flexible CG''
(FCG)~\cite{doi:10.1137/S1064827597323415} incurs an overhead that
corresponds to keeping the last residual vector in memory and computing an
additional vector operation and \dotp product. The benefits are that the
iterative method can handle a flexible (non-constant)
preconditioner~\cite{notay}, which is needed when applying a preconditioner in
reduced precision.

\begin{figure}[t]
\lstinputlisting[language=Matlab]{code/pcg.m}
\caption
[Algorithmic formulation (in MATLAB) of the PCG method]
{Algorithmic formulation (in MATLAB) of the PCG method.
For a problem of size $n$ containing $n_z$ nonzero elements in the system matrix
stored in CSR format, ignoring the preconditioner application, each PCG
iteration requires $(14n+n_z)\cdot\fpd+(n_z+n)\cdot\inti$ memory transactions.}
\label{2017-adaptive-block-jacobi:fig:codepcg}
\end{figure}

\begin{figure}[t]
\lstinputlisting[language=Matlab]{code/fcg.m}
\caption
[Algorithmic formulation (in MATLAB) of the FCG method]
{Algorithmic formulation (in MATLAB) of the FCG method.
For a problem of size $n$ containing $n_z$ nonzero elements in the system matrix
stored in CSR format, ignoring the preconditioner application, each FCG
iteration requires $(21n+n_z)\cdot\fpd+(n_z+n)\cdot\inti$ memory transactions.}
\label{2017-adaptive-block-jacobi:fig:codefcg}
\end{figure}


Obviously, with a constant preconditioner, $r_{k}^T z_{k+1} = 0$, i.e. both
formulas~\eqref{2017-adaptive-block-jacobi:eq:fr} and~\eqref{2017-adaptive-block-jacobi:eq:pr} are identical.
For $r_{k}^T z_{k+1} \neq 0$, the Polak-Ribi\`{e}re formula specifies a locally 
optimal search
direction, which means that the convergence rate of this method will not be 
slower than that of a locally optimal steepest descent
method~\cite{doi:10.1137/060675290}. We complement the preconditioned CG method,
based on the Fletcher-Reeves formula shown in Figure~\ref{2017-adaptive-block-jacobi:fig:codepcg}, with the
flexible conjugate gradient (FCG) method based on the Polak-Ribi\`{e}re formula 
in Figure~\ref{2017-adaptive-block-jacobi:fig:codefcg}. The two codes differ only in lines 6--8 
(computation of {\tt \small gamma\_new} and additional recurrence for vector 
{\tt t}), which results in $7n$ additional memory accesses.  A faster 
preconditioner application 
(i.e., using reduced precision arithmetic operations in the actual 
preconditioner
application) could barely compensate for this overhead.

In our approach, we store the preconditioner at reduced precision, 
but we convert the data to double precision right after reading it from
memory and before invoking the arithmetic computations of the preconditioner
application. Hence, although stored at a reduced precision, the preconditioner
itself remains constant across all iterations. This strategy does
introduce some overhead in terms of converting the preconditioner data to
double precision and using double precision in all arithmetic operations, but it
comes with the benefit of using the Fletcher-Reeves formula~\eqref{2017-adaptive-block-jacobi:eq:fr} for
the orthogonalization step, which results in the more attractive (in terms 
of memory) standard PCG solver.

