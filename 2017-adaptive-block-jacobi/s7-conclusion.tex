\section{Concluding Remarks and Future Work} \label{2017-adaptive-block-jacobi:sec:summary}

We proposed and validated a strategy to reduce the data transfer volume in
a block--Jacobi preconditioner. Concretely, our technique individually selects
an appropriate precision format to store the distinct blocks of the
preconditioner based on their characteristics but performs all arithmetic
(including the generation of the preconditioner) in \fpd. We note
that the condition numbers can be obtained cheaply as our preconditioner is
based on explicit inversion of the diagonal blocks.
Furthermore, the overhead from selecting the appropriate storage format in the
preconditioner setup can easily be amortized by the reduced cost of the
preconditioner application in the solver iterations.

Our experimental simulation using Octave on an Intel architecture shows that, in
most cases, storing a block--Jacobi preconditioner in \fps has only a
mild impact on the preconditioner quality. On the other hand, the reckless use 
of \fph to
store a block-Jacobi preconditioner fails in most cases and is therefore not 
recommended. The adaptive precision block--Jacobi preconditioner basically
matches the convergence rate of the conventional double precision preconditioner
in all cases and automatically adapts the precision to be used on an individual
basis. As a result, the adaptive precision preconditioner can decide to store
some of the blocks at precisions even less than \fps, thereby outperforming a 
fixed precision strategy that relies on only a single precision in terms of data
transfers and, consequently, energy consumption.

As part of our future work, we plan to investigate the effect of using
other, non {\sc ieee}-compliant data formats in the adaptive block--Jacobi
preconditioner, prioritizing the exponent range at the cost of reducing the
bits dedicated to the mantissa. In this endeavor, we expect to reduce the
problems with underflows/overflows while maintaining the ``balancing''
properties of the preconditioner. Furthermore, we will also develop a
practical implementation of the adaptive precision block--Jacobi using {\sc
  ieee} formats with 16, 32, and 64 bits for modern GPUs.

